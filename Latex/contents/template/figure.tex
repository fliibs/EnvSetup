% !TEX root = ../../main.tex

\clearpage
\section{Figure}

\subsection{插图}

插图功能是利用 \TeX{} 的特定编译程序提供的机制实现的,不同的编译程序支持不同的图
形方式。有的同学可能听说“\LaTeX{} 只支持 EPS”,事实上这种说法是不准确的。\XeTeX{}
可以很方便地插入 EPS、PDF、PNG、JPEG 格式的图片。

一般图形都是处在浮动环境中。之所以称为浮动是指最终排版效果图形的位置不一定与源文
件中的位置对应,这也是刚使用 \LaTeX{} 同学可能遇到的问题。如果要强制固定浮动图形
的位置,请使用 \pkg{float} 宏包,它提供了 \texttt{[H]} 参数。

\subsubsection{单个图形}

如图~\ref{fig.google} 所示。

\begin{figure}[ht]
    \centering
    \includegraphics[width=7cm]{./figure/google.png} \\
      1.google figure 2.different color % 对图片进行解释
    \caption{Main name 2} %最终文档中希望显示的图片标题
    {Stay hungry, stay foolish.}
   \label{fig.google}
\end{figure}

\subsubsection{多个图形}
简单插入多个图形的例子如图~\ref{fig:SRR} 所示。这两个水平并列放置的子图共用一个
图形计数器,没有各自的子图题。

\begin{figure}[ht]
  \centering
  \includegraphics[height=2cm]{figure/badge-blue.pdf}
  \hspace{1cm}
  \includegraphics[height=2cm]{figure/badge-blue.pdf}
  \caption{English caption}
  \label{fig:SRR}
\end{figure}

如果多个图形相互独立,并不共用一个图形计数器,那么用 \texttt{minipage} 或者
\texttt{parbox} 就可以,如图~\ref{fig:parallel1} 与图~\ref{fig:parallel2}。

\begin{figure}[ht]
\begin{minipage}{0.48\textwidth}
  \centering
  \includegraphics[height=1.5cm]{figure/name-blue.pdf}
  \caption{并排第一个图}
  \label{fig:parallel1}
\end{minipage}\hfill
\begin{minipage}{0.48\textwidth}
  \centering
  \includegraphics[height=1.5cm]{figure/name-blue.pdf}
  \caption{并排第二个图}
  \label{fig:parallel2}
\end{minipage}
\end{figure}