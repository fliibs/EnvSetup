% !TEX root = ../../main.tex

% \flbssetup{
%     title = {\huge\textbf{Project \textit{A2} IP \textit{XXX}\\Hardware Design Document}},

%     author = {name},
% }

% 字体选择
% \tiny, \scriptsize, \footnotesize, \small, \normalsize, \large, \Large, \LARGE, \huge, \Huge
% \textbf 加粗
% \textit 斜体

%层次结构
% \part {...} %Level -1
% \chapter {...} %Level 0
% \section {...} %Level 1
% \subsection {...} %Level 2
% \subsubsection {...} %Level 3
% \paragraph {...} %Level 4
% \subparagraph {...} %Level 5

% 页面大小
\usepackage{geometry}
\geometry{a4paper,scale=0.7}


\usepackage{caption}
\captionsetup{font={scriptsize}}
\CTEXsetup[format={\Large\bfseries}]{section} % section左对齐
\CTEXsetup[format={\large\bfseries}]{subsection} % section左对齐
\CTEXsetup[format={\normalsize\bfseries}]{subsubsection} % section左对齐
\CTEXsetup[format={\small\bfseries}]{paragraph} % section左对齐

\usepackage{indentfirst}
\setlength{\parindent}{2em} %缩进的距离(2em 表示缩进 2 个字符位置)

% 字体颜色
\usepackage{color}
\usepackage[table,xcdraw]{xcolor}
% \textcolor{red/blue/green/black/white/cyan/magenta/yellow}{text}

% 字体调整
\usepackage{fontspec}
\setmainfont{Times New Roman}


% \indent 首行缩进
% \noindent 不首行缩进
% \raggedright \raggedleft \centering 对齐设置

\usepackage{graphicx}   % 插入图片用到的宏包
\usepackage{multirow}   % 插入表格用到的宏包
% \usepackage{ctex}       % 支持中文
\usepackage{amsmath}    % 数学公式支持


% 缩进
\usepackage{indentfirst} 
\setlength{\parindent}{2em} %2em代表首行缩进两个字符

% 页脚页码
% \pagestyle{empty}
\usepackage{fancyhdr}            %使用fancyhdr包
 
% 脚注格式
\usepackage[perpage,bottom,hang]{footmisc}

% 定义图片文件目录与扩展名
\graphicspath{{figures/}}
\DeclareGraphicsExtensions{.pdf,.eps,.png,.jpg,.jpeg}
% 定义图片的序号
\usepackage{chngcntr}
\counterwithin{figure}{section}

% 确定浮动对象的位置,可以使用 [H],强制将浮动对象放到这里(可能效果很差)
% \usepackage{float}

% 固定宽度的表格
% \usepackage{tabularx}

% 使用三线表:toprule,midrule,bottomrule。
\usepackage{booktabs}

% 表格中支持跨行
\usepackage{multirow}

% 表格中数字按小数点对齐
\usepackage{dcolumn}
\newcolumntype{d}[1]{D{.}{.}{#1}}

% 使用长表格
\usepackage{longtable}

% 附带脚注的表格
\usepackage{threeparttable}

% 附带脚注的长表格
\usepackage{threeparttablex}

% 算法环境宏包
\usepackage[ruled,vlined,linesnumbered]{algorithm2e}
% \usepackage{algorithm, algorithmicx, algpseudocode}

% 代码环境宏包
\usepackage{listings}
\lstnewenvironment{codeblock}[1][]%
  {\lstset{style=lstStyleCode,#1}}{}

% 直立体数学符号
\providecommand{\dd}{\mathop{}\!\mathrm{d}}
\providecommand{\ee}{\mathrm{e}}
\providecommand{\ii}{\mathrm{i}}
\providecommand{\jj}{\mathrm{j}}

% 国际单位制宏包
\usepackage{siunitx}[=v2]

% 定理环境宏包
\usepackage{ntheorem}
% \usepackage{amsthm}

% 绘图宏包
\usepackage{tikz}
\usetikzlibrary{shapes.geometric, arrows}

% 一些文档中用到的 logo
\usepackage{hologo}
\providecommand{\XeTeX}{\hologo{XeTeX}}
\providecommand{\BibLaTeX}{\textsc{Bib}\LaTeX}

% 借用 ltxdoc 里面的几个命令方便写文档
\DeclareRobustCommand\cs[1]{\texttt{\char`\\#1}}
\providecommand\pkg[1]{{\sffamily#1}}

% 自定义命令

% E-mail
\newcommand{\email}[1]{\href{mailto:#1}{\texttt{#1}}}

% hyperref 宏包在最后调用
\usepackage{hyperref}






