% !TEX root = ../../main.tex

% \flbssetup{
%     title = {\huge\textbf{Project \textit{A2} IP \textit{XXX}\\Hardware Design Document}},

%     author = {name},
% }

% 字体选择
% \tiny, \scriptsize, \footnotesize, \small, \normalsize, \large, \Large, \LARGE, \huge, \Huge
% \textbf 加粗
% \textit 斜体

%层次结构
% \part {...} %Level -1
% \chapter {...} %Level 0
% \section {...} %Level 1
% \subsection {...} %Level 2
% \subsubsection {...} %Level 3
% \paragraph {...} %Level 4
% \subparagraph {...} %Level 5

% 页面大小
\usepackage{geometry}
\geometry{a4paper, left=31.8mm, right=31.8mm,top=30.4mm, bottom=30.4mm}


\usepackage{caption}
\captionsetup{font={normalsize}}
\CTEXsetup[format={\erhao\bfseries}]{section} % section左对齐
\CTEXsetup[format={\xiaoer\bfseries}]{subsection} % section左对齐
\CTEXsetup[format={\sanhao\bfseries}]{subsubsection} % section左对齐
\CTEXsetup[format={\xiaosan\bfseries}]{paragraph} % section左对齐
\fontsize{12.0pt}{12pt}\selectfont  
% \titleformat{\chapter}[hang]{\centering\LARGE\bfseries}{\chaptername}{1em}{}
     

% 目录调整
\usepackage{titletoc}
\contentsmargin{0pt}
\renewcommand\contentspage{\thecontentspage}
\dottedcontents{section}[2.3em]{}{2.3em}{5pt}
\dottedcontents{subsection}[5.5em]{}{3.2em}{5pt}


\usepackage{indentfirst}
\setlength{\parindent}{2em} %缩进的距离(2em 表示缩进 2 个字符位置)

% 字体颜色
\usepackage{color}
\usepackage[table,xcdraw]{xcolor}
% \textcolor{red/blue/green/black/white/cyan/magenta/yellow}{text}
\definecolor{blue}{rgb}{0,0.48,0.8}
\definecolor{red}{HTML}{FF0800}
\definecolor{grey}{HTML}{808080}

% 字体调整
\usepackage{fontspec}
\setmainfont{Times New Roman}
\newcommand{\yihao}{\fontsize{26pt}{36pt}\selectfont}           % 一号, 1.4 倍行距
\newcommand{\erhao}{\fontsize{22pt}{28pt}\selectfont}          % 二号, 1.25倍行距
\newcommand{\xiaoer}{\fontsize{18pt}{18pt}\selectfont}          % 小二, 单倍行距
\newcommand{\sanhao}{\fontsize{16pt}{24pt}\selectfont}        % 三号, 1.5倍行距
\newcommand{\xiaosan}{\fontsize{15pt}{22pt}\selectfont}        % 小三, 1.5倍行距
\newcommand{\sihao}{\fontsize{14pt}{21pt}\selectfont}            % 四号, 1.5 倍行距
\newcommand{\banxiaosi}{\fontsize{13pt}{19.5pt}\selectfont}    % 半小四, 1.5倍行距
\newcommand{\xiaosi}{\fontsize{12pt}{18pt}\selectfont}            % 小四, 1.5倍行距
\newcommand{\dawuhao}{\fontsize{11pt}{11pt}\selectfont}           % 大五号, 单倍行距
\newcommand{\wuhao}{\fontsize{10.5pt}{15.75pt}\selectfont}        % 五号, 单倍行距


% \indent 首行缩进
% \noindent 不首行缩进
% \raggedright \raggedleft \centering 对齐设置

\usepackage{graphicx}   % 插入图片用到的宏包
\usepackage{multirow}   % 插入表格用到的宏包
% \usepackage{ctex}       % 支持中文
\usepackage{amsmath}    % 数学公式支持


% 缩进
\usepackage{indentfirst} 
\setlength{\parindent}{2em} %2em代表首行缩进两个字符

% 页脚页码
% \pagestyle{empty}
\usepackage{fancyhdr}            %使用fancyhdr包
 
% 脚注格式
\usepackage[perpage,bottom,hang]{footmisc}

% 定义图片文件目录与扩展名
\graphicspath{{figures/}}
\DeclareGraphicsExtensions{.pdf,.eps,.png,.jpg,.jpeg}
% 定义图片的序号
\usepackage{chngcntr}
\counterwithin{figure}{section}
\counterwithin{table}{section}


% item设置
\usepackage{enumitem}
\setenumerate[1]{itemsep=0pt,partopsep=0pt,parsep=\parskip,topsep=5pt}
\setitemize[1]{itemsep=0pt,partopsep=0pt,parsep=\parskip,topsep=5pt}
\setdescription{itemsep=0pt,partopsep=0pt,parsep=\parskip,topsep=5pt}

% 确定浮动对象的位置,可以使用 [H],强制将浮动对象放到这里(可能效果很差)
% \usepackage{float}

% 固定宽度的表格
\usepackage{tabularx}
\usepackage{array} 
\renewcommand\arraystretch{0.5} 

% 使用三线表:toprule,midrule,bottomrule。
\usepackage{booktabs}

% 表格中支持跨行
\usepackage{multirow}

% 表格中数字按小数点对齐
\usepackage{dcolumn}
\newcolumntype{d}[1]{D{.}{.}{#1}}

% 使用长表格
\usepackage{longtable}

% 附带脚注的表格
\usepackage{threeparttable}

% 附带脚注的长表格
\usepackage{threeparttablex}

% 算法环境宏包
\usepackage[ruled,vlined,linesnumbered]{algorithm2e}
% \usepackage{algorithm, algorithmicx, algpseudocode}

% 代码环境宏包
\usepackage{listings}
\lstnewenvironment{codeblock}[1][]%
  {\lstset{style=lstStyleCode,#1}}{}

% 直立体数学符号
\providecommand{\dd}{\mathop{}\!\mathrm{d}}
\providecommand{\ee}{\mathrm{e}}
\providecommand{\ii}{\mathrm{i}}
\providecommand{\jj}{\mathrm{j}}

% 国际单位制宏包
\usepackage{siunitx}[=v2]

% 定理环境宏包
\usepackage{ntheorem}
% \usepackage{amsthm}

% 绘图宏包
\usepackage{tikz}
\usetikzlibrary{shapes.geometric, arrows}

% 一些文档中用到的 logo
\usepackage{hologo}
\providecommand{\XeTeX}{\hologo{XeTeX}}
\providecommand{\BibLaTeX}{\textsc{Bib}\LaTeX}

% 借用 ltxdoc 里面的几个命令方便写文档
\DeclareRobustCommand\cs[1]{\texttt{\char`\\#1}}
\providecommand\pkg[1]{{\sffamily#1}}

% 自定义命令

% E-mail
\newcommand{\email}[1]{\href{mailto:#1}{\texttt{#1}}}

% hyperref 宏包在最后调用
\usepackage{hyperref}






